\documentclass[11pt,a4paper]{article}

\usepackage{mathptmx} % Times New Roman font
\usepackage[margin=1in]{geometry} % Set page margins
\usepackage{setspace} % Set line spacing
\usepackage{color}

\renewcommand{\familydefault}{\rmdefault} % Set default font to Times New Roman

% Personal information
\newcommand{\name}{Shan Gao}
\newcommand{\address}{604 W 8th Ave, Mesa, AZ 85210}
\newcommand{\phone}{(602) 693-6743}
\newcommand{\email}{sgao52@asu.edu}
\newcommand{\website}{https://github.com/sgao52}

% Section formatting
\usepackage{titlesec}
% \titleformat{\section}{\large\bfseries}{\thesection}{1em}{}  % original one
\titleformat{\section}{
  \vspace{-12pt}\scshape\raggedright\large
}{}{0em}{}[\color{black}\titlerule \vspace{3pt}]

% Margin format
\addtolength{\oddsidemargin}{-0.5in}
\addtolength{\evensidemargin}{-0.5in}
\addtolength{\textwidth}{1in}
\addtolength{\topmargin}{-.5in}
\addtolength{\textheight}{1.0in}

\raggedbottom
\raggedright
\setlength{\tabcolsep}{0in}

% Entry formatting
\newcommand{\entry}[4]{\noindent\textbf{#1} \hfill #2\\ \textit{#3} \hfill #4 \vspace{7 pt}\\}
\newcommand{\Rexp}[3]{\noindent\textbf{#1} \hfill #2\\ \textit{#3} \vspace{-9 pt}\\}
\newcommand{\proj}[3]{\noindent\textbf{#1} \hfill #2\\ \hfill #3 \vspace{7 pt}\\}
\newcommand{\conf}[2]{#1 \hfill #2 \vspace{7 pt}\\}
\newcommand{\train}[4]{#1 \hfill #2\\ \emph{#3} \hfill #4 \vspace{7 pt}\\}

%% Item formatting
\newcommand{\resumeItem}[2]{
  \item\small{
    \textbf{#1}{: #2 \vspace{-6pt}}
  }
}
\newcommand{\resumeItemListStart}{\begin{itemize}}
\newcommand{\resumeItemListEnd}{\end{itemize}\vspace{-5pt}}

\pagestyle{empty} % remove page number
% Document content
\begin{document}

% Heading
\begin{center}
    {\LARGE \textbf{\name}}\\
    \vspace{0.2in}
    \address\\
    \vspace{0.05in}
    \phone \ \textbullet \ \email \ \textbullet \ \website
\end{center}

% Education
\section{Education}
\entry{Geographic information Systems}{Arizona State University}{Master of Advanced Study}{2021--2022}
\entry{Parks and Recreation Management}{Arizona State University}{Bachelor of Science}{2017--2021}
\entry{Human Geography and Urban Planning}{Hainan University}{Bachelor of Science}{2017--2021}



% Academic Course work
\section{Academic Coursework}
\begin{itemize}
    \item Introduction to Geographic Information Systems (A)
    \item Intermediate GIS (A)
    \item Spatial Statistics (A)
    \item Implementation in Corporate and Public Sectors (B)
    \item GIS Project Presentation (B)
    \item Programming the GIS Environment (A)
    \item GIS Technologies (A+)
    \item GIS for Business (A)
    \item GIS for the Internet (A)
\vspace{7pt}
\end{itemize}


% Research experience
\section{Research Experience}
\Rexp{Student Intern}{2021--2022}{Center for Global Discovery and Conservation Science}
    \resumeItemListStart
        \resumeItem{Data Processing}{Based on satellite imagery modify the Mangrove China 2018 data and created 849 polygons as
        mangrove core areas for data verification.}
        \resumeItem{Classification}{Use Random Forest Supervised Classification method to classify the Hainan Island mangrove forest
        with Landsat 8 satellite imagery}
        \resumeItem{ArcGIS Model}{Combine mangrove 2018 data and satellite imagery to recreate the general distribution of
        mangrove with tool of reclassify, majority filter, and boundary clean tool}
    \resumeItemListEnd
\vspace{7pt}
\Rexp{Student Assistant}{2020--2020}{China University of Geoscience}
    \resumeItemListStart
        \resumeItem{Information Collection \& Screen}{Research specific cases in Danxia mountain, Mogao Grottoes, The Grand Canyon, and Yellow Stone National Park in literature, comparing the legislation and motivation system between China and U.S. volunteering system.}
        \resumeItem{Data Analyze Related to World Heritage}{Collect and analyze data related to World Heritage volunteer initiatives and provide a report to demonstrate growth opportunity of China World Heritage volunteer system.}
    \resumeItemListEnd
\vspace{7pt}
% Skills
\section{Skills}
\begin{itemize}
    \item Programming languages: Python, HTML, MATLAB, SQL, LaTeX, JavaScript
    \item GIS/Remote Sensing software: ArcGIS, Metashape, Google Earth Engine, AutoCAD
    \item Other Tools \& softwares: SQL Server Management Studio, Azure, Ubuntu, VOSviewer
\end{itemize}

% Projects
\section{Projects}
\proj{Capstone Project - Monitoring Central Arizona Lakes}{2022}{Created timelapse by Google Earth Engine from 2010 to 2021 and Palmer Drought Severity Index to find out the
proper year to analyze. Classify Mormon Lake, Lynx Lake, and Watson Lake using ISO Cluster Unsupervised Classification, Support Vector Machine Supervised Classification, and U-Net Deep Learning Classification methods in ArcGIS. Compare different classification methods on accuracy level and visual effect level, and calculate the estimated time of processing.}
\proj{Tempe Public Art Installation Project}{2021}{Propose several methods and data to find the potential art installation according to the core concept of economic
equity, social equity, and racial equity. Reference the offenses analysis and land use data, locate three vacant areas in Tempe, and suggest building a landmark.Implement the geodatabase, well-designed maps, analysis, web app, presentation, and executive summary for the project.}
\proj{Graduation Project - Monitoring Mangrove Change in Nature Reserve}{2021}{Classify the mangrove area in Dongzhaigang Nature Reserve, using Landsat 8 data from 2014 to 2020 on Google Earth Engine. Bring about NDVI \& mangrove area change curve from 2013 to 2020, also provides protection recommendations for local authority based on the mangrove shrinking fact and interview.}

% Meetings&Conferences
\section{Meetings \& Conferences}
\conf{American Association of Geographer 2023 Annual Meeting}{2023}
\conf{UAV Vegetation Remote Sensing Technology and Application Salon}{2023}
\conf{The Colorado River at the Compact's Centennial}{2022}


% Professional memberships
\section{Professional Memberships}
\conf{American Assiociation of Geographers}{2022 - Now}
% Certifications and Training
\section{Certifications \& Training}
\train{Advanced Computing Techniques Training}{2023}{Arizona State University}{}
\train{FAA Part 107 Remote Pilot Certificate}{2023-2025}{Federal Aviation Administration}{}
\train{Beginner's Guide to Research Computing Training}{2022}{Arizona State University}{}
\train{The Third Quantitative Remote Sensing Graduation Certificate}{2021}{Wuhan University}{}
% Referees
% \section{Referees}


\end{document}
